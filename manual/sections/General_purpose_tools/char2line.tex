\section{Program goose-char2line}
The \texttt{goose-char2line} splits a sequence into lines, creating an output sequence which has a char for each line.

For help type:
\begin{lstlisting}
./goose-char2line -h
\end{lstlisting}
In the following subsections, we explain the input and output paramters.

\subsection*{Input parameters}

The \texttt{goose-char2line} program needs two streams for the computation,
namely the input and output standard. The input stream is a sequence file.\\
The attribution is given according to:
\begin{lstlisting}
Usage: ./goose-char2line [options] [[--] args]
   or: ./goose-char2line [options]

It splits a sequence into lines, creating an output sequence which has a char for each line.

    -h, --help        show this help message and exit

Basic options
    < input.seq       Input sequence file (stdin)
    > output.seq      Output sequence file (stdout)

Example: ./goose-char2line < input.seq > output.seq
\end{lstlisting}
An example on such an input file is:
\begin{lstlisting}
ACAAGACGGCCTCCTGCTGCTGCTGCTCTCCGGGGCCACGGCCCTGGAGGGTCCACCGCTGCCCTGCTGCCATTGTCCCC
GGCCCCACCTAAGGAAAAGCAGCCTCCTGACTTTCCTCGCTTGGGCCGAGACAGCGAGCATATGCAGGAAGCGGCAGGAA
GTGGTTTGAGTGGACCTCCGGGCCCCTCATAGGAGAGGAAGCTCGGGAGGTGGCCAGGCGGCAGGAAGCAGGCCAGTGCC
GCGAATCCGCGCGCCGGGACAGAATCTCCTGCAAAGCCCTGCAGGAACTTCTTCTGGAAGACCTTCTCCACCCCCCCAGC
TAAAACCTCACCCATGAATGCTCACGCAAGTTTAATTACAGACCTGAAACAAGATGCCATTGTCCCCCGGCCTCCTGCTG
CTGCTGCTCTCCGGGGCCACGGCCACCGCTGCCCTGCCCCTGGAGGGTGGCCCCACCGGCCGAGACAGCGAGCATATGCA
GGAAGCGGCAGGAATAAGGAAAAGCAGCCTCCTGACTTTCCTCGCTTGGTGGTTTGAGTGGACCTCCCAGGCCAGTGCCG
GGCCCCTCATAGGAGAGGAAGCTCGGGAGGTGGCCAGGCGGCAGGAAGGCGCACCCCCCCAGCAATCCGCGCGCCGGGAC
AGAATGCCCTGCAGGAACTTCTTCTGGAAGACCTTCTCCTCCTGCAAATAAAACCTCACCCATGAATGCTCACGCAAGTT
TAATTACAGACCTGAA
\end{lstlisting}

\subsection*{Output}
The output of the \texttt{goose-char2line} program is a group sequence splited by \textbackslash n foreach character.\\
An example, for the input, is:
\begin{lstlisting}
A
C
A
A
G
A
C
G
G
C
C
T
C
C
T
G
C
T
G
C
T
...
\end{lstlisting}