\section{Program gto\char`_fastq\char`_variation\char`_visual}
The \texttt{gto\char`_fastq\char`_variation\char`_visual} depicts the regions of singularity using the output from \texttt{gto\char`_fastq\char`_variation\char`_filter} into an SVG image.\\
For help type:
\begin{lstlisting}
./gto_fastq_variation_visual -h
\end{lstlisting}
In the following subsections, we explain the input and output parameters.

\subsection*{Input parameters}

The \texttt{gto\char`_fastq\char`_variation\char`_visual} program needs the output of \texttt{gto\char`_fastq\char`_variation\char`_filter} to compute.\\
The attribution is given according to:
\begin{lstlisting}
Usage: ./gto_fastq_variation_visual <OPTIONS>... [FILE]:<...>
./gto_fastq_variation_visual: visualize relative singularity regions.
                                                     
  -v                       verbose mode,             
  -a                       about CHESTER,            
  -e <value>               enlarge painted regions,  
                                                     
  [tFile1]:<tFile2>:<...>  target file(s).           
                                                     
Report bugs to <{pratas,raquelsilva,ap,pjf}@ua.pt>. 
\end{lstlisting}
An example of such an input file is:
\begin{lstlisting}
#132#132
30:60
90:130
\end{lstlisting}

\subsection*{Output}
The output of the \texttt{gto\char`_fastq\char`_variation\char`_visual} program is an SVG plot with the maps.\\
Figure~\ref{fig:gtoFastqVariationVisual} presents the plot obtained using the input above.

\begin{figure}[!h]
\centering
\includegraphics[scale=0.6]{./images/gto_fastq_variation_visual.png}
\caption{\texttt{gto\char`_fastq\char`_variation\char`_visual} execution plot.}
\label{fig:gtoFastqVariationVisual}
\end{figure}