\section{Program gto\char`_amino\char`_acid\char`_from\char`_fasta}
The \texttt{gto\char`_amino\char`_acid\char`_from\char`_fasta} converts DNA sequences in FASTA or Multi-FASTA file format to an amino acid sequence.\\
For help type:
\begin{lstlisting}
./gto_amino_acid_from_fasta -h
\end{lstlisting}
In the following subsections, we explain the input and output parameters.

\subsection*{Input parameters}

The \texttt{gto\char`_amino\char`_acid\char`_from\char`_fasta} program needs two streams for the computation, namely the input and output standard. The input stream is a FASTA or Multi-FASTA file.\\
The attribution is given according to:
\begin{lstlisting}
Usage: ../../bin/gto_amino_acid_from_fasta [options] [[--] args]
   or: ../../bin/gto_amino_acid_from_fasta [options]

It converts FASTA or Multi-FASTA file format to an amino acid sequence (translation).

    -h, --help            Show this help message and exit

Basic options
    < input.mfasta        Input FASTA or Multi-FASTA file format (stdin)
    > output.prot         Output amino acid sequence file (stdout)

Optional
    -f, --frame=<int>     Translation codon frame (1, 2 or 3)

Example: ../../bin/gto_amino_acid_from_fasta < input.mfasta > output.prot
\end{lstlisting}
An example of such an input file is:
\begin{lstlisting}
>AB000264 |acc=AB000264|descr=Homo sapiens mRNA 
ACAAGACGGCCTCCTGCTGCTGCTGCTCTCCGGGGCCACGGCCCTGGAGGGTCCACCGCTGCCCTGCTGCCATTGTCCC
CGGCCCCACCTAAGGAAAAGCAGCCTCCTGACTTTCCTCGCTTGGGCCGAGACAGCGAGCATATGCAGGAAGCGGCAGG
AAGTGGTTTGAGTGGACCTCCGGGCCCCTCATAGGAGAGGAAGCTCGGGAGGTGGCCAGGCGGCAGGAAGCAGGCCAGT
GCCGCGAATCCGCGCGCCGGGACAGAATCTCCTGCAAAGCCCTGCAGGAACTTCTTCTGGAAGACCTTCTCCACCCCCC
CAGCTAAAACCTCACCCATGAATGCTCACGCAAGTTTAATTACAGACCTGAA
>AB000263 |acc=AB000263|descr=Homo sapiens mRNA 
ACAAGATGCCATTGTCCCCCGGCCTCCTGCTGCTGCTGCTCTCCGGGGCCACGGCCACCGCTGCCCTGCCCCTGGAGGG
TGGCCCCACCGGCCGAGACAGCGAGCATATGCAGGAAGCGGCAGGAATAAGGAAAAGCAGCCTCCTGACTTTCCTCGCT
TGGTGGTTTGAGTGGACCTCCCAGGCCAGTGCCGGGCCCCTCATAGGAGAGGAAGCTCGGGAGGTGGCCAGGCGGCAGG
AAGGCGCACCCCCCCAGCAATCCGCGCGCCGGGACAGAATGCCCTGCAGGAACTTCTTCTGGAAGACCTTCTCCTCCTG
CAAATAAAACCTCACCCATGAATGCTCACGCAAGTTTAATTACAGACCTGAA
\end{lstlisting}

\subsection*{Output}

The output of the \texttt{gto\char`_amino\char`_acid\char`_from\char`_fasta} program is an amino acid sequence.\\
Using the input above, an output example of this is:
\begin{lstlisting}
TRRPPAAAALRGHGPGGSTAALLPLSPAPPKEKQPPDFPRLGRDSEHMQEAAGSGLSGPPGPS-ERKLGRWPGGRKQAS
AANPRAGTESPAKPCRNFFWKTFSTPPAKTSPMNAHASLITDLTRCHCPPASCCCCSPGPRPPLPCPWRVAPPAETASI
CRKRQE-GKAAS-LSSLGGLSGPPRPVPGPS-ERKLGRWPGGRKAHPPSNPRAGTECPAGTSSGRPSPPANKTSPMNAH
ASLITDL
\end{lstlisting}
